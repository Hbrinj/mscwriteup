% !TEX root = ./main.tex
\graphicspath{ {images/design/} }
\sctn{Design}
``A modern paradox is that it's simpler to create complex interfaces because it's so complex to simplify them'' – P{\"a}r Almqvist, This perfectly describes the modern interactive age where users are exposed to interfaces which have large development teams that still get it wrong! Clear and concise description of the program is the number one priority of a user interface. Too much information on display and the user will be cognitively overwhelmed; too little and they will be bewildered \cite{design_principles}. The user interface for KoMoDo at present is overwhelming and out of date. Modern developers are used to development applications which share common features. The commonality between applications makes them inherently intuitive.\\\\
%
Organisation of the visual elements is something else missing in KoMoDo. Grouping similar functionality provides structure to the UI while reducing the cognitive load. Furthermore, consistency with the design of other applications can help users to feel more in control of their environment\cite{design_principles_two}.
%Talk about some design principles
\ssctn{KoMoDo UI Breakdown}
\fig{ \centering\img{0.4}{oldkomodo-anno.png} }{Current KoMoDo UI}{currentUI}
KoMoDo's dated UI leaves a lot to be desired. The UI requires not just a fresh coat of paint but also needs some elements to be merged. UIs in applications built today have a lot in common. The consistent feel makes them intuitive. The commonality of UI elements is what KoMoDo is missing. Before getting to the design work its important to understand where KoMoDo's UI fails.
%
\sssctn{The Control Bar}
KoMoDo's control bar, labelled as \itl{1} at the top in figure \ref{fig: currentUI}, is an attempt to cram a large amount of functionality into the given space so that everything is at the users fingertips. Some of the buttons can be merged into one atomic function, for instance, the \itl{compile} and \itl{load} operations can be merged into one distinct button without the need for a single line edit next to it. The buttons already open a file dialog so the user won't be missing anything. Furthermore, instead of consistently loading the source using either \itl{compile} or \itl{load}, a \itl{reload} button will be provided which reloads the current source file, compiling it if need be.\\\\
%
The program control buttons such as \itl{run} and \itl{continue} can be merged given they perform the same operation. \itl{Reset} and \itl{stop} can also be merged and a new button, \itl{pause}, can be introduced which will pause the execution of the current code. Furthermore, to bring the controls in line with modern IDEs and debuggers the button text could be replaced with images\footnote{People always say a picture is worth a thousand words}.\\\\
%
Finally, it will be a good idea to separate the bar into sections with each section providing a set of operations which work in a similar way. Doing this may also help decrease the overall size of the control bar and make it more approachable for new users.

\sssctn{The Register Window}
The register pane, labelled as \itl{2} in figure \ref{fig: currentUI} is very simple and doesn't require a great deal of change. In particular it would simply need a way to notify the user that a register has changed. This notification can allow users to better follow the changes, as they single-step their code.

\sssctn{The Memory Window}
The memory window \itl{3}, does exactly what it is supposed to by providing a breakdown of the loaded source code. Highlighting the syntax should make it easier to use for new users, allowing them to easily differentiate between the different parts of the machine code. Additionally, bringing the breakpoints more inline with other IDEs will make it more intuitive for the user. Furthermore, as the main focus of the window, it should take most of the available space. Removing the second memory window should provide this necessary space.

\sssctn{The Second Memory Window with Editing}
This memory window is a waste of screen real estate, editing the code in a text editor and using the reload button makes this element obsolete. It would be best to remove this, doing so can allow more of the memory to be exposed to the user.
\ssctn{Proposed Design}
The initial UI design underwent an iterative development cycle show in figure \ref{fig: iterative-design}, these designs where further evaluated among my peers. The feedback provided was taken and used for the final design shown in figure \ref{fig: proposedUI}.
\fig{
\centering
\begin{tikzpicture}[>=latex]
	%elements
	\node [rounded corners=1mm]at (0,0)   (start) [rectangle,draw=black!100, minimum width=10mm, minimum height=5mm] {Start};

	\node [] at ($(start.south)+(0cm,-1cm)$) (needs) [rectangle,draw=black!100, minimum width=30mm, minimum height=5mm] {Determine needs};
	\node [] at ($(needs.south)+(0cm,-1cm)$) (build) [rectangle,draw=black!100, minimum width=30mm, minimum height=5mm] {Build design};
	\node [] at ($(build.south)+(0cm,-1cm)$) (eval) [rectangle,draw=black!100, minimum width=30mm, minimum height=5mm] {Evaluate design};
	\node [draw, diamond, aspect=2] at ($(eval.south)+(0cm,-1cm)$) (des) [draw=black!100, minimum width=10mm, minimum height=5mm] {};
	\node [rounded corners=1mm]at ($(des.south)+(0cm,-1cm)$)   (end) [rectangle,draw=black!100, minimum width=10mm, minimum height=5mm] {End};
	%text
	\draw [->] (start.south) -- (needs.north);
	\draw [->] (needs.south) -- (build.north);
	\draw [->] (build.south) -- (eval.north);
	\draw [->] (eval.south) -- (des.north);
	\draw [->] (des.south) -- (end.north);
	\draw [->] (des.east) -- ($(des.east) + (2cm,0cm)$) -- ($(needs.east) + (1cm,0cm)$) node [pos=0.5,right,font=\footnotesize] {Iterate} -- (needs.east);


\end{tikzpicture}
}{Iterative design process}{iterative-design}
\pagebreak
%
The proposed interface show in figure \ref{fig: proposedUI}, in the Appendix, implements the required changes to KoMoDo. Designing the interface was done with the core design principles\cite{design_principles} in mind. The Interface is simple yet effective, providing users with the necessary operations to perform their task. A particular effort was made to group similar elements as well as, reduce the number of overall elements, so as to reduce cognitive overload.\\\\
%
This, of course is a rough sketch with the eventual UI potentially being different, but it provides a good overview of the proposed changes.
%Implements many of the changes talked about in the previous section with the sole aim of keeping it clean
%This is a rough sketch of the changes
%Actual interface may vary
