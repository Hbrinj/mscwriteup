% !TEX root = ./main.tex
\graphicspath{ {images/requirements/} }

\sctn{Requirements}
This section is dedicated to capturing the requirements of potential users within the scope of KoMoDos educational use. Personas will be created to conceptualise the requirements required to achieve the desired results.\\\\
%
System requirements should be malleable transforming over time as use cases and users change. Change in the requirements over time allows the end product to better suit the software needs. Using the agile development methodology allows for a reactive planning and development process, focusing on early delivery and continuous development.\\\\
%
The functional requirements of KoMoDo++ will be derived from the persona use cases.

\ssctn{Use case}
To fully realise the required features of KoMoDo++, two separate personas have been created. The requirements specification will be created based on the use case of each persona. Use cases provide useful insight into the requirements of a particular type of user based on their interactions with the system. Developing use cases helps model the context of the system as well as providing in-depth analysis into a users needs, aiding in a more refined set of requirements.\\\\
%
Personas are a tool to capture the context of use. Therefore, contrasting personas have been created to better reflect interactions with KoMoDo++. Moreover, personas provide insight which can help make informed architectural design decisions.
%
\sssctn{Personas}
\bld{Mike}\\
Mike is a first year computer scientist and one of his first set of modules involves learning about low level hardware/software. Mike has made his own website before and did so using a text editor, he thought syntax highlighting made it easier for him to read the code. Mike has a particular interest in UI/UX design. Mike is a windows user and has not used Linux before, he thinks he will eventually use Linux. Mike sees that KoMoDo++ is one of the tools used to work on the coursework and so chooses to download it.\\\\
%
\bld{Lucy}\\
Lucy is a third year computer scientist working on her dissertation. She has decided to create her own low level language along with its own compiler. In her first year she learnt about the 32bit ARM chip architecture but shes forgotten some of the intricacies. She has decided she needs to brush up on the chip and would find it useful if she could see a breakdown of changes that occur at each step. Lucy also uses a mac as her daily driver. She downloads KoMoDo++ for the advanced features it provides.
%
\sssctn{Scenarios}
\bld{Scenario 1: Mike}\\
Mike has just been given his first assignment which asks him to implement a simple multiplication function using ARM32. He writes some of the code and spins up KoMoDo++ for the first time on his windows laptop. Mike takes note that the basic version of KoMoDo++ is simple and inviting to use. Consequently, he finds its intuitive to get started and so immediately loads up his code. Mike starts stepping through the code to see how it transforms over time. He likes that the ARM instructions in the memory window use syntax highlighting. He also likes that there is a break down of what an instruction does.  Mike carries on slowly implementing his solution and finds KoMoDo++ easy to use.\\\\
%
\bld{Scenario 2: Lucy}\\
Lucy begins to design her machine language, she uses a simple program written in ARM32 as the basis for her design. She has not used KoMoDo++ yet and she notes that upon opening KoMoDo++ it was good that there was a break down of its features. She starts by stepping through the code in the advanced view and takes note about the flags that change. She is particularly pleased with the code breakdown as it gives her insight into what the command does on the fly. Lucy carries on slowly building up her new language, referring to the decisions made for ARM32.
