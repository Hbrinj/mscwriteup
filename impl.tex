% !TEX root = ./main.tex
\graphicspath{ {images/impl/} }
\sctn{Implementation}
  \ssctn{Complications}
  Unfortunatly, KoMoDo turned out to be far more complex than originally thought\footnote{More of a pandoras box really}. Some of the issues encountered are discussed further within this section. The conclusion outlines the rest of the development plan for KoMoDo.
    \sssctn{User-Interface}
    When KoMoDo was conceived, \itl{Glade}, the visual UI creator did not exist. Consequently, all of KoMoDo's UI is hardcoded. Therefore, each element is individually packed into containers, shown to the user and even manipulated. The tight coupling between the UI and KoMoDo itself poses porting challenges. Additionally, a large collection of features are deprecated in later iterations of \itl{GTK}.\\\\
    %
    Furthermore, later iterations of \itl{GTK} allow developers to theme their application using CSS styling, with each UI element being part of a CSS selector. With \itl{GTK1} this isn't quite possible, it requires the use of theming software such as \itl{QTCurve} which, changes the UI of all \itl{GTK}-built applications.\\\\
    %
    Finally, one of the features discussed was syntax highlighting for the memory display, with \itl{GTK2} onwards the treeview requires that the data structure contain a class which knows how to draw the column data onto the view. \itl{GTK2} would have made implementing this feature far simpler, allowing each instruction to be highlighted accordingly. This is not possible in \itl{GTK1}, the CLIST is a rigid widget which draws only text/pixmaps onto the screen. Furthermore, colouring is limited to rows or columns as opposed to individual characters. A possible way to over come this issue is to generate pixmaps which represent the machine instructions, drawing them onto the list instead.
    %GTK2 allows themeing via CSS, GTK1 does not, possibility to theme using qtcurve
    %Syntax highlighting requires ability to render own text
    \sssctn{Porting to GTK3}
    The original idea was to port KoMoDo to a recent stable version of \itl{GTK}. Porting to \itl{GTK3} would have prolonged KoMoDo's shelf life. With proper maintenance KoMoDo on \itl{GTK3} could have been kept relevant for years to come. Unfortunatly, \itl{GTK3} deprecates features found in \itl{GTK2} and completely removes features that were originally deprecated in \itl{GTK2}. Moreover, \itl{GTK3} changes how widgets and various logic is used.\\\\
    %
    At this point it was evident that \itl{GTK3} is a far stretch, with so many changes to the program flow it would be better to reimplement KoMoDo from the ground up, something which took the original creators far longer than is permitted for the scope of this project.
    %instant issues due to major API overhaul
    %clists dropped in favour of treeviews
    \sssctn{Porting to GTK2}
    \itl{GTK2} was the next target version after \itl{GTK3} proved to be far more perplexing. \itl{GTK2} allows for deprecated features to be enabled, this would allow for iterative changes to the UI without breaking everything.\\\\
    %
    To begin with, \itl{Glade} was used to implement the UI change proposed in the design chapter. \itl{Glade} for \itl{GTK2} came with its own problems. \itl{Glade} would often crash (tested on multiple devices) the crash would cause inconsistencies within the XML interface file. Consequently, the interface needed to be remade. This was a lesson quickly learnt, due to the nature of this work version control was already in use. Consequently, regular backups were performed after minor changes.\\\\
    %
    Using the \itl{Glade} interface file is straightforward and requires only a few \itl{GTK} function calls. Problems arose when connecting the glade UI to the old logic, \itl{Glade} didn't allow the use of deprecated features. Therefore, it was necessary to port some of the code over to use the new treeview implementation. In a broad sense this was successful, but in actual fact it introduced buggy logic between the old \itl{GTK1} code and the new \itl{GTK2} code. Additionally, the documentation and coding style were far from desireable, it was evident from the sparse comments asking `what does this do' that the majority of KoMoDo was implemented by a single person. Furthermore, many of KoMoDo's features needed to be disabled so that the new UI could be integrated, something that would be detrimental to KoMoDo.
    % There exists a flag which allows for deprecated features to be enabled
    % The new user interface is written in glade
    % clsists exist but are part of the deprecated features
    % new uses treeview, using treeview causes alot of features to be dropped due to logic differences
    % alot of unkowns still exist documentation is lax,
    \sssctn{Conclusion}
    After battling with \itl{GTK2}, \itl{GTK3} and \itl{Glade} it was evident that KoMoDo should be rebuilt from the ground up, a task which should either be made open source or a group project. Further deliberation and meetings with Professor David Brailsford fortified this conclusion. Most of the \itl{Jimulator} and parts of \itl{Chump} could stay relatively unchanged, but KoMoDo would need to be completely rewritten. This would bring about greater opportunity to do things correctly and remove the tight coupling between the UI and logic.\\\\
    %
    Consequently, to at least improve KoMoDo's current UI, work will be carried on using \itl{GTK1}, building features onto KoMoDo `as is' while refining the UI. Reducing some of the buttons in the action bar should bring KoMoDo more in line with the design of other IDEs, making it more intuitive to use. Working the breakpoints into the memory display will allow for more intuitive use, all this can be achieved while still maintaining the complex memory window. Moreover, some sort of memory visualization which would display occupied sectors could be useful to programmers. Finally, converting some of the buttons into pictures will help to convey their purpose.
    %Porting is not feasible, would be better to re-implement KoMoDo from the ground up
    %including logic.
    %Working with the current system is better, can provide consistancy within the code
    %UI can still be improved, reducing clutter

  \ssctn{Language and Platform Decisions}
  Both the language and platform that KoMoDo is being developed on are pre-determined due to the existing software. KoMoDo (as discussed in the frameworks chapter) uses \itl{GTK+} 1.2 and \itl{GLib} 1.2 . Due to the complications with porting no further changes will be made to the API version for \itl{GTK} and \itl{GLib}.\\\\
  %
  As for language, KoMoDo was written in pure C and so any changes to the logic will also be made using C.\\\\
  %
  Furthermore, the primary platform that KoMoDo needs to run on is *nix. It will be developed primarily on a *nix machine and then if possible, on other platforms. The choice of framework is the determining factor when attempting to run on multiple platforms. In KoMoDo's case it is heavily dependent on \itl{GTK} and \itl{GLib}. Consequently, KoMoDo would require a full rewrite to make it platform agnostic.
  %predetermined due to the nature of the project
  %using GTK+ 1.2 and Glib 1.2
  \ssctn{Tools and Technologies}
  \sssctn{Building}
  KoMoDo is built using \itl{gcc} but uses \itl{Autoconf} and \itl{Make} to configure the correct library paths and includes. The external libraries are dynamically linked with KoMoDo, this means that the user must first install the KoMoDo dependencies.

  \sssctn{Code}
  Code is written in a text editor, in particular \itl{Vim}. \itl{Vim} is a barebones editor with no minimal functionality. It is fast, reliable and comes with a whole host of useful features.\\\\
  KoMoDo is being developed on an Intel NUC (Next Unit of Computing) running Arch Linux. As the main driver the NUC is fast and runs a barebones Linux build which is perfect for development.

  \sssctn{Backups}
  GitHub hosts Git which is a distributed version control system and the primary form of backup. Along with GitHub, backups are regularly made to Google drive and Dropbox, with the GitHub backup always being the most up-to-date. Git branches are used to test and work on features which are in progress, this keeps the Master branch functional. Finally, minor changes are often committed so that a full log of the progress is maintained.

  \sssctn{Progress Management}
  A Kanban is used to track features through the inception/implementation/testing/complete phases. Sprints occur regularly, typically taking place from Wednesday through to Saturday. The sprints are for both the documentation and code. The overall progress is monitored using a Gantt chart.
  %Vim for text editing
  %built on an Intel Nuc
  %
  \ssctn{Project Management}
  Project management is an important part of the software development pipeline. But software projects are subject to failure due to several factors \cite{software_failure}:
  \itm{
    \item Unrealistic project goals
    \item Badly defined system requirements
    \item Poor project management
    \item Sloppy development practices
  }
%
  Software developers now are far more conscious of development demands and can better determine outlandish goals. They acquire these via university, accelerator courses or as a by-product of their work as a developer. Some, may not understand the gravity of their decisions until such time as project implementation begins. This is all part of the experience of becoming a developer.\\\\
  %
  Early development of the requirements, and research, somewhat aided in the decisions made in this project. However, there was no prior experience porting code to more relevant APIs. The realisation that porting was out of the question put the project at risk. However, it has been a valuable lesson in dealing with legacy systems.\\\\
  %
  Due to the diligent time planning the errors in judgement were inconsequential to the overall software development. Planning for issues is all part of using a Gantt chart. Furthermore, using the Agile approach meant that both the software and write-up were quickly progressed to reflect the change in requirements. The use of a Kanban aided in micro progress updates as the project was being developed.\\\\
  %
  Finally, regular meetings with everyone involved in the project meant they were kept up-to-date. This provided a great deal of flexibility and trust to be established by all parties involved.

  \ssctn{Implementation of Software}
  Software development was carried out with the best coding practices in mind. Additionally, it was important to develop code which was inline with the current style, this would enable future developers to more readily become accustomed to the already in-place coding practices.
  \sssctn{Prototype}
  The prototype can be thought of as the current version of KoMoDo. Modifications to the UI and functionality are derived from the intended design changes. Additionally, new features which were discussed as part of the requirements are now part of the final working product. The actual duration of the prototyping phase extends from the initial research into KoMoDo's inner workings all the way up to the start of the implementation. \\\\
  %
  The breadth of the prototyping phase is due to the nature of the project, it is crucial to understand how the components already existent in KoMoDo interact and their implications. Furthermore, it was important to fully develop an idea of what KoMoDo already had in place. There are many features in KoMoDo that, due to the convoluted UI, are obfuscated.
  \sssctn{Final product}
  Changes to KoMoDo were made using scarce \itl{GTK1} documentation and by using the existing code as reference. The logic used to make the changes may well not be optimal, but given the resources, it was the best possible.\\\\
  \bld{\medium{User Interface\\\\}}
  % Reworked the UI
  One of the primary areas of improvement, the user interface, was reworked first. Emphasis was put into replacing unnecessary text with images, thereby reducing the clutter. KoMoDo's own hard-coded UI was the baseline for understanding \itl{GTK1} because the documentation was so sparse.
  %
  \fig{\centering\img{0.5}{kmdmain}}{New KoMoDo UI}{kmdnew}
  %
  \pagebreak
  During the design stages another issue that was identified was the density of the tool bar and the obscurity of the program control `multi-step' and `walk'. Steps were taken to reduce the visual impact of the toolbar and increase the clarity of both `multi-step' and `walk'.\\\\
  %
  The revamped visuals follow the standard design patterns for similar programs; providing a more intuitive experience.\\\\
  %
  \bld{\medium{Breakpoints}}
  \fig{\centering\img{0.5}{kmdmemory}}{Breakpoints in memory}{kmdmemory}
  % Reworked the breakpoint display in the main memory window
  Breakpoints in the original KoMoDo were obscure and rarely used by new students. To increase their usage a particular effort was made to bring their styling inline with other common debuggers. Additionally, their explicit usage is now detailed in the KoMoDo beginners guide.\\\\
  %

  \bld{\medium{Register Display\\\\}}
  % Reworked the register display to make it clearer what change had occured
  \fig{\centering\img{0.5}{kmdreg}}{Memory notifications}{kmdregister}
  When debugging an application intricate details can often be missed. To alleviate this problem the register pane was reworked to highlight changes which occur during the program's execution. Highlighting the changes provides users with additional information so that they are made aware of an instruction's exact actions.\\\\
  %
  \bld{\medium{Loading Source Files\\\\}}
  % Reworked how code is opened
  KoMoDo's original UI used a manually driven two-step process by which, source files were converted to a \func{KMD} format and then finally loaded. The potential reason for using this two-step process was that \func{KMD} files could be loaded without the need for an initial compilation step. However, this method of opening files was arduous and counter intuitive. The reworked version now does away with this two-step process opening either the source or \func{KMD} file. Additionally, a reload button was added which reloads the previously loaded source file. The reload button allows users to quickly reload code that they have edited in a text editor.\\\\
  %
  \bld{\medium{KoMoDo Installation\\\\}}
  Parts of the build process have been updated to easily setup a base system. This base system now provides the \itl{AASM} used to compile the source file and the compile script; which KoMoDo looks for to compile sources. The addition of these files provides users with the ability to begin writing and debugging ARM32 straight after installation\footnote{provided they can get gtk and glib installed}. Additionally, KoMoDo's configuration files are now stored in the standard Linux configuration path.\\\\
  % Reworked requirements for installation
  %
  \bld{\medium{Automatic Config Creation\\\\}}
  Upon attempting to launch KoMoDo for the first time unless \func{kmd -c} was issued prior to launching the actual application KoMoDo would not start. The requirement to manually force config creation degrades the user experience. To allay the issue KoMoDo will now automatically create the configuration file if one did not exist.\\\\
  %
  \bld{\medium{Memory Visualisation\\\\}}
  For students and power users to better understand the memory impact of their code memory visualisation was introduced. The visualisation highlights areas where code exists. Additionally, users can use this memory visualisation to draw images using their code much like Figure \ref{fig: memvis}.\\\\
  \fig{\centering\img{0.5}{memvis}}{Memory Visualisation}{memvis}
  %
  \bld{\medium{Bug Fix: Orphaned Jimulator\\\\}}
  A rather common bug which has struck countless users of KoMoDo is the orphaned \itl{Jimulator} process. The process remains running even when KoMoDo had crashed, eventually leading to an army of orphaned processes. The fix for this bug relies on the Linux kernel to notify and kill the orphaned process.\\\\
  %
  \bld{\medium{Bug Fix: Jimulator Incorrectly Resolved on Start-up\\\\}}
  After installation, when attempting to run KoMoDo, if the \itl{Jimulator} was not in the current working directory KoMoDo would halt. This bug occurs due to the way KoMoDo resolved the \itl{Jimulator} binary, it would only look in the current working directory. To fix this, KoMoDo now starts the \itl{Jimulator} by resolving it using the bash environment. This won't work if the folder containing the \itl{Jimulator} is not in the \func{\$PATH} variable.\\\\
  % Jimulator being resolved incorrectly
  %
  \bld{\medium{Bug Fix: Phantom Second Memory Scrollbar\\\\}}
  Upon resizing the window, if the frame became too narrow a phantom scrollbar would appear with no apparent use. This fix was trivial, it would seem that through the countless iterations of KoMoDo, between its initial inception and the addition of a range based scrolling system, the developer must have forgotten the type of container the memory window was added to. Directly adding the memory window to the parent container fixed the phantom scrollbar problem.\\\\
  %
  \bld{\medium{Additions: Virtual Machine Image\\\\}}
  To provide alternative access to KoMoDo offline, a VM image running Centos 7 was created via \itl{Virtual Box}. This image has a pre-built version of KoMoDo set-up and running. Users must install \itl{Virtual Box} and run the image to begin using KoMoDo. Using KoMoDo this way removes the need for individuals to locate the legacy dependencies \itl{GTK1} and \itl{GLib-1.2}. Additionally, it can be run on multiple operating systems. Its explicit usage is detailed within the beginners guide.\\\\
  %
  \bld{\medium{Additions: Beginners Guide\\\\}}
  In an effort to further demystify KoMoDo an extensive beginners guide was developed. The beginners guide details the features within KoMoDo and how to use them. The guide also provides details about installing KoMoDo on a Linux machine, detailing any complications.
  % Bug fixes:
  % Jimulator being orphaned
  % Installation missing required files for KoMoDo to use arm

  % Implemented memory visualisation
  % usage additions:
  % Created stand-alone VM which students can aquire and use
