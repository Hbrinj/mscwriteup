% !TEX root = ./main.tex
\sctn{Background}
KoMoDo is an interactive screen-display tool, using X-Windows on the Unix/GNU Linux architecture. It was originally designed by Charlie Brej of the Dept. of Computer Science, University of Manchester. The majority of the current maintenance is handled by Jim Garside -- a senior lecturer at the University of Manchester.\\\\
%
The early history of KoMoDo is hazy but it was built using \itl{GLib} and \itl{GTK 1.2} which both date back to 1999.\\\\
%
KoModo is used to teach students the low-level architecture of modern computers. The University of Nottingham uses it as a tool in the first year for the core module G51CSF. It provides an environment in which students can learn about machine code and learn the complexities of instruction design.\\\\
%
KoMoDo teaches students to think about the programs they write in a low-level manner, which is close to the `bare metal'. This is in contrast to the more abstract style of a typical high-level language.\\\\
%
KoMoDo, in the context of a teaching aid, is primarily run in emulation mode. The back-end emulator running as a process, accessed via a two-way Unix pipe, is called the \itl{Jimulator}. The \itl{Jimulator} is the driving force behind KoMoDo.\\\\
%
KoMoDo also uses a disassembler called \itl{Chump} (if rendered upside down in an unknown font it spells `dump'). \itl{Chump} is used to convert the resulting assembler instructions back to something that KoMoDo can use and display to the user.\\\\
%
Appreciation for KoMoDo runs deep. Although first year students do not generally realise the importance of KoMoDo until much later on in their degree course, it is nevertheless  important to keep it alive and fully functional.
%
\sctn{Aims and Objectives}
\ssctn{Aims}
The aim of this project is to redesign KoMoDo and provide it with much needed enhancements and features. The enhancements will come in several forms; a revamped user interface, a better user experience and syntax-highlighted ARM instructions. KoMoDo's portability to platforms other than Linux is also on the agenda. For the duration of this project Linux will remain the primary target platform. Students receive exposure to Linux only via the use of KoMoDo, as such, it is one of the primary objectives that KoMoDo performs fully on Linux systems.
\ssctn{Objectives}
\itm{
	\item Dissect KoMoDo \itm{
			\item Discover how KoMoDo communicates with the \itl{Jimulator}
			\item Discover how memory is shared between the \itl{Jimulator} and KoMoDo
			\item Discover how \itl{Chump} is used to show disassembled ARM instructions
		}
	\item Future proof KoMoDo \itm{
			\item Use a well established framework such as \itl{QT}
			\item Ensure libraries used are easily changed and or updated
			\item Develop a modern UI and improved UX
		}
	\item Portability and Platforms \itm{
			\item Ensure KoMoDo++ works on Linux distributions
			\item Make moves towards ensuring KoMoDo++'s use on multiple platforms
		}
	\item Stretch goals \itm{
			\item Convert the \itl{Jimulator} to work independently of the platform
			\item Convert \itl{Chump} to work independently of the platform
			\item Convert the assembler to work independently of the platform
		}
}
%Portability of KoMoDo will be achieved using QT, a framework designed with cross compilation in mind. QT provides a meta-object compiler which enriches the C++ language providing cross compilation support. QTs enrichments also mean that the model-view-controller pattern is further simplified. QTs solution to messaging between views is to allow users to subscribe to events. Furthermore said events contact the subscriber when an event is triggered.
% modernize komodo by
	% changing the UI
	% Improving the UX
	% Adding new features
		% syntax highlighting
	% future proofing with regards to portability

%Ensure it works on linux becase currently thats the only exposure students recieve within the university
% primary objectives
	% Ensure it runs on linux
	% Disect KoMoDo
		% Find out how KoMoDo uses the jimulator
		% Work on how the memeory should be presented

% Stretch Goals
	% Modify AASM to work on any platform
	% Modify the jimulator to work on any platform

\sctn{External Interests}
The University of Manchester where KoMoDo was originally conceived still uses KoMoDo as a teaching aid. Several other universities including the University of Nottingham are also still using it to teach students the fundamentals of computer architecture. It is in the interest of the aforementioned universities that KoMoDo is able to be used by students regardless of platform and library dependencies.
% Working with Manchester University to improve this learning tool
% A key tool used in teaching students about the inner workings of modern computers
	% ALlows students to experiment with the theory they learn
	% Changes the ways students percieve their applications, allows them to think more atomically
	% the atomic thinking is the bases of many courses such as concurrency and optimization

\sctn {Work Plan}
To fully realise the challenges of this project it is best to explore the main components for a UI/UX overhaul. First, it is important to understand how KoMoDo operates currently, how its UI was built and to identify any dependencies on other software. Secondly, we need to understand what frameworks are available and the feasibility of their use for this project. \\\\
%
Once the research has been complete, via use cases, a system requirements specification can be derived to fully encapsulate KoMoDo++'s desired functionality. Moreover, this project will be carried out using the Agile methodology. Working using Agile means that the project aims will change to meet the desired outcomes.\\\\
%
A breakdown of the proposed time line is provided via a Gantt chart. Refer to figure \ref{fig: gantt} in the Appendix.
