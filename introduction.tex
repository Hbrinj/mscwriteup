\sctn{Background}
KoMoDo is an interactive screen-display tool, using X-Windows on the Unix/GNU Linux architecture. It was originally designed by Charlie Brej of the Dept. of Computer Science, University of Manchester. The majority of the current maintenance is handled by Jim Garside -- a lecturer at the University of Manchester.\\\\
%
The early history of KoMoDo is hazy but it was built using GLib and GTK 1.2 which both date back to 1999.\\\\
%
KoModo is used to teach students the low-level architecture of modern computers. The University of Nottingham uses it as a tool in the first year for the core module G51CSF. It provides an environment in which students can learn about machine code and learn the complexities of instruction design.\\\\ 
%
KoMoDo teaches students to think about the programs they write in a low-level manner, which is close to the `bare metal'. This is in contrast to the more abstract style of a typical high-level language.\\\\
%
KoMoDo, in the context of a teaching aid, is primarily run in emulation mode. The back-end emulator running as a process and accessed via a two-way Unix pipe, is called the jimulator. The jimulator is the driving force behind KoMoDo.\\\\
%
KoMoDo also uses a disassembler called chump (if rendered upside down in an unknown font it spells `dump'). Chump is used to convert the resulting assembler instructions back to something that KoMoDo can use and display to the user.\\\\
%
Appreciation for KoMoDo runs deep. Although first year students do not generally realise the importance of KoMoDo until much later on in their degree course, it is nevertheless  important to keep it alive and fully functional.
%
\sctn{Aims and Objectives}
\ssctn{Aims}
The aim of this project is to redesign KoMoDo and provide it with much needed enhancements and features. The enhancements will come in several forms; a revamped user interface, a better user experience and syntax highlighted ARM instructions. KoMoDo's portability to platforms other than Linux is also on the agenda. For the duration of this project Linux will remain the primary target platform. Students receive exposure to Linux only via the use of KoMoDo so it is a primary goal to ensure full functionality in the Linux ecosystem.
\ssctn{Objectives}
\itm{
	\item Dissect KoMoDo \itm{
			\item Discover how KoMoDo communicates with the jimulator
			\item Discover how memory is shared between the jimulator and KoMoDo
			\item Discover how Chump is used to show disassembled ARM instructions
		}
	\item Future proof KoMoDo \itm{
			\item Use a well established framework such as QT
			\item Ensure libraries used are easily changed and or updated
			\item Develop a modern UI and improved UX
		}
	\item Portability and Platforms \itm{
			\item Ensure KoMoDo++ works on Linux distributions
			\item Make moves towards ensuring KoMoDo++'s use on multiple platforms 
		}
	\item Stretch goals \itm{
			\item Convert the jimulator to work independently of the platform
			\item Convert Chump to work independently of the platform
			\item Convert the assembler to work independently of the platform
		}
}
%Portability of KoMoDo will be achieved using QT, a framework designed with cross compilation in mind. QT provides a meta-object compiler which enriches the C++ language providing cross compilation support. QTs enrichments also mean that the model-view-controller pattern is further simplified. QTs solution to messaging between views is to allow users to subscribe to events. Furthermore said events contact the subscriber when an event is triggered.
% modernize komodo by
	% changing the UI
	% Improving the UX
	% Adding new features
		% syntax highlighting
	% future proofing with regards to portability

%Ensure it works on linux becase currently thats the only exposure students recieve within the university
% primary objectives
	% Ensure it runs on linux
	% Disect KoMoDo
		% Find out how KoMoDo uses the jimulator
		% Work on how the memeory should be presented

% Stretch Goals
	% Modify AASM to work on any platform
	% Modify the jimulator to work on any platform

\sctn{External Interests}
The university of Manchester where KoMoDo was originally conceived still uses KoMoDo as a teaching aid. Several other universities including the university of Nottingham are also still using it to teach students the fundamentals. It is in the interest of the aforementioned universities that KoMoDo is able to be used by students regardless of platform and library dependencies.
% Working with Manchester University to improve this learning tool
% A key tool used in teaching students about the inner workings of modern computers
	% ALlows students to experiment with the theory they learn
	% Changes the ways students percieve their applications, allows them to think more atomically
	% the atomic thinking is the bases of many courses such as concurrency and optimization

\sctn {Work Plan}
The plan is to first and foremost update the UI, UX and ensure KoMoDo++ works on Linux. Finally if time permitting the stretch goals will be tackled one by one. Refer to figure: \ref{fig: gantt} on page \pageref{fig: gantt} for the Gantt chart relating to this project.
